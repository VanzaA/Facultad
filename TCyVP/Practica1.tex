
\documentclass{article}

\usepackage[margin=1.5in]{geometry} % Please keep the margins at 1.5 so that there is space for grader comments.
\usepackage{amsmath,amsthm,amssymb,hyperref}
\usepackage[utf8]{inputenc}

\title{Máquinas de Turing \\ Jerarquía de la Computabilidad.}
\centering
\begin{document}
\newenvironment{solution}{\begin{proof}[Solution]}{\end{proof}}
\maketitle

\large 

\begin{center}

{\Large Agustin Vanzato \\
Federio Gasquez} % Replace "Author's Name" with your name
\end{center}
\vspace{0.05in}

\begin{enumerate}

\item \textbf{ Responder brevemente los siguientes incisos:}

\begin{enumerate}
\item \textbf{¿Qué es un problema (computacional) de decisión? ¿Es el tipo de problema más general que se puede formular?}
Un problema de decisión es un problema en donde las respuestas posibles son "sí" o "no".
\item \textbf{¿Qué cadenas integran el lenguaje aceptado por una MT?}
El conjunto de las cadenas aceptadas por la MT M es el lenguaje aceptado o reconocido por M, y se denota con L(M). Considerando la visión de
MT M calculadora, sólo cuando M se detiene en un estado q $\in$ F debe tenerse en cuenta
el contenido final de la cinta, es decir la cadena de salida (o simplemente la salida).
\item \textbf{En la clase teórica 1 se hace referencia al problema de satisfactibilidad de las fórmulas booleanas (se da como ejemplo la fórmula $\varphi = (x_1 \vee x_2) \wedge (x_3 \vee x_4)$ y la asignación A = (V, F, V, V)).\\ Formular las tres formas del problema, teniendo en cuenta las tres visiones de MT consideradas: calculadora, aceptadora o reconocedora y generadora.}

\item \textbf{¿Qué postula la Tesis de Church-Turing?}

la conjetura conocida como Tesis de Church-Turing, indica que todo lo computable puede ser llevado a
cabo por una máquina de Turing

\item \textbf{¿Cuándo dos MT son equivalentes? ¿Cuándo dos modelos de MT son equivalentes?}
Dos máquinas de turing son equivalentes cuando reconocen el mismo lenguaje. Dos modelos de
máquina de turing son equivalentes cuando poseen el mismo poder de cómputo, es decir a partir de una
se puede simular la otra y viceversa.
\item \textbf{¿En qué difiere un lenguaje recursivo de un lenguaje recursivamente numerable no recursivo?}

Un lenguaje es recursivamente numerable si y sólo si existe una MT que lo reconoce.
Es decir, si L es el conjunto de todos los lenguajes (cada uno integrado por cadenas
finitas de símbolos pertenecientes a un alfabeto universal $\sum$), sólo los lenguajes
recursivamente numerables de L son reconocibles por una MT (por esto es que a los
problemas de decisión asociados se los conoce como computables). La clase de los
lenguajes recursivamente numerables se denomina RE (por recursively enumerable
languages). El nombre se debe a que las cadenas de estos lenguajes se pueden
enumerar. De esta manera, dado L $\in$ RE, si M es una MT tal que L(M) = L, se cumple
para toda cadena w de $\sum^*$ que:
\begin{itemize}
\item Si w $\in$ L, entonces M a partir de w se detiene en su estado $q_A$ .
\item Si w $\notin$ L, entonces M a partir de w se detiene en su estado q R o no se detiene.
\end{itemize}
Se define que un lenguaje es recursivo si y sólo si existe una MT M que lo reconoce y que se detiene cualquiera sea su entrada. La clase
de los lenguajes recursivos se denomina R. A los problemas de decisión asociados se los
conoce como decidibles, porque las MT que los resuelven pueden justamente decidir,
cualquiera sea la instancia, si es positiva o negativa. Ahora, dado L $\in$ R, si M es una
MT tal que L(M) = L, se cumple para toda cadena w de $\sum^*$ que:
\begin{itemize}
\item Si w $\in$ L, entonces M a partir de w se detiene en su estado q A .
\item Si w $\notin$ L, entonces M a partir de w se detiene en su estado q R .
\end{itemize}
\item \textbf{¿En qué difiere un lenguaje recursivamente numerable de uno que no lo es?}
\item \textbf{Probar que R $\subseteq RE \subseteq \mathcal{L}.$}
\item \textbf{¿Cuándo un lenguaje está en la clase CO-RE? ¿Puede un lenguaje estar al mismo tiempo en la clase RE y en la clase CO-RE? ¿Para todo lenguaje de la clase CO-RE existe una MT que lo acepta?}
\item \textbf{Justificar por qué los lenguajes $\sum$ y $\emptyset$ son recursivos.}
\item \textbf{Si $L \subseteq \sum^*$, ¿se cumple que L $\in$ R?}
\item \textbf{Justificar por qué un lenguaje finito es recursivo.}
\item \textbf{Justificar por qué si L1 $\in$ CO-RE y L2 $\in$ CO-RE, entonces (L1 $\wedge$ L2) $\in$ CO-RE}
\end{enumerate}

\item \textbf{ Dado el alfabeto $\sum$ = \{a, b, c\}:}

\begin{enumerate}

    \item \textbf{ Obtener el lenguaje $\sum^*$ y el conjunto de partes del subconjunto de $\sum^*$ con cadenas de a lo
sumo dos símbolos. ¿Cuál es el cardinal (o tamaño) de este último conjunto?  }

\item \textbf{Dado el lenguaje $L = {a^n b^n c^n | n \geq 0}$, 
obtener la intersección 
$\sum^* \cap$ L, la unión $\sum^* \cup$ L, el
complemento de L respecto de $\sum^*$, 
y la concatenación $\sum^*$ . L.}

\end{enumerate}

\item \textbf{ Construir una MT (puede tener varias cintas) que acepte de la manera más eficiente
posible el lenguaje $L = \{a^n b^n c^n | n \geq 0\}$. Plantear primero la idea general.}

\item \textbf{Explicar (informal pero claramente) cómo simular una MT por otra que en un paso
no pueda simultáneamente modificar un símbolo y moverse.}

\item \textbf{Explicar (informal pero claramente) cómo simular una MT por otra que no tenga el
movimiento S (es decir el no movimiento).}

\item \textbf{Sea USAT el lenguaje de las fórmulas booleanas satisfactibles con exactamente una
asignación de valores de verdad. P.ej. $x_1 \wedge x_2$ pertenece a USAT, mientras que $(x_1 \wedge x_2) \vee x_3$ no.
Indicar, justificando la respuesta, si la siguiente MTN acepta USAT: }

\begin{enumerate}
\item \textbf{Si la fórmula de entrada no es correcta sintácticamente, rechaza.}
\item \textbf{ Genera no determinísticamente una asignación A, y si A no satisface la fórmula, rechaza.}
\item \textbf{Genera no determinísticamente una asignación A’ $\neq$ A. Si A’ no satisface la fórmula, acepta,
y si A’ la satisface, rechaza.}
\end{enumerate}

\textbf{Ayuda: Considerar p.ej. el caso en que la fórmula tiene dos asignaciones que la satisfacen.
}

\item \textbf{Considerando el Lema 4 estudiado en la Clase Teórica 2 $(R = RE \cap CO-RE)$:}

\begin{enumerate}
\item \textbf{ Construir la MT M.}
\item \textbf{Probar la correctitud de la construcción.} 
\end{enumerate}

\item \textbf{Sean $L_1$ y $L_2$ dos lenguajes recursivamente numerables de números naturales
representados en notación unaria (por ejemplo, el número 5 se representa con 11111). Probar
que también es recursivamente numerable el lenguaje L = \{x $\mid$ x es un número natural
representado en notación unaria, y existen y, z, tales que y + z = x, con y $\in L_1$, z $\in L_2$\}.\\}
\textbf{Ayuda: la prueba es similar a la de la clausura de RE con respecto a la concatenación. }
 
\item \textbf{ Dada una MT M1 con $\sum$ = }

\begin{enumerate}
\item \textbf{ Construir una MT $M_2$ que determine si L($M_1$) tiene al menos una cadena.}
\item \textbf{ ¿Se puede construir además una MT $M_3$ para determinar si L($M_1$) tiene a lo sumo una
cadena? Justificar.
}
\end{enumerate}

\textbf{Ayuda para la parte (1): Si L(M1) tiene al menos una cadena, entonces existe al menos una
cadena w de unos y ceros, de tamaño n, tal que M1 a partir de w acepta en k pasos. Teniendo
en cuenta esto, pensar cómo M2 podría simular M1 considerando todas las cadenas de unos y
ceros hasta encontrar eventualmente una que M1 acepte (¡cuidándose de los casos en que M1
entre en loop!)}

\end{enumerate}

\end{document}


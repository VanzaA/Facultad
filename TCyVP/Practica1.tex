
\documentclass{article}

\usepackage[margin=1.5in]{geometry} % Please keep the margins at 1.5 so that there is space for grader comments.
\usepackage{amsmath,amsthm,amssymb,hyperref}
\usepackage[utf8]{inputenc}

\title{Máquinas de Turing \\ Jerarquía de la Computabilidad.}
\centering
\begin{document}
\newenvironment{solution}{\begin{proof}[Solution]}{\end{proof}}
\maketitle

\large 

\begin{center}

{\Large Agustin Vanzato \\
Federio Gasquez} % Replace "Author's Name" with your name
\end{center}
\vspace{0.05in}

\begin{enumerate}

\item \textbf{Ejercicio 1. Responder brevemente los siguientes incisos:}

\begin{enumerate}
\item \textbf{¿Qué es un problema (computacional) de decisión? ¿Es el tipo de problema más general que se puede formular?}
\item \textbf{¿Qué cadenas integran el lenguaje aceptado por una MT?}
\item \textbf{En la clase teórica 1 se hace referencia al problema de satisfactibilidad de las fórmulas booleanas (se da como ejemplo la fórmula $\varphi = (x_1 \vee x_2) \wedge (x_3 \vee x_4)$ y la asignación A = (V, F, V, V)).\\ Formular las tres formas del problema, teniendo en cuenta las tres visiones de MT consideradas: calculadora, aceptadora o reconocedora y generadora.}

\item \textbf{¿Qué postula la Tesis de Church-Turing?}
\item \textbf{¿Cuándo dos MT son equivalentes? ¿Cuándo dos modelos de MT son equivalentes?}
\item \textbf{¿En qué difiere un lenguaje recursivo de un lenguaje recursivamente numerable no recursivo?}
\item \textbf{¿En qué difiere un lenguaje recursivamente numerable de uno que no lo es?}
\item \textbf{Probar que R $\subseteq RE \subseteq \mathcal{L}.$}
\item \textbf{¿Cuándo un lenguaje está en la clase CO-RE? ¿Puede un lenguaje estar al mismo tiempo en la clase RE y en la clase CO-RE? ¿Para todo lenguaje de la clase CO-RE existe una MT que lo acepta?}
\item \textbf{Justificar por qué los lenguajes $\sum$ y $\emptyset$ son recursivos.}
\item \textbf{Si $L \subseteq \sum^*$, ¿se cumple que L $\in$ R?}
\item \textbf{Justificar por qué un lenguaje finito es recursivo.}
\item \textbf{Justificar por qué si L1 $\in$ CO-RE y L2 $\in$ CO-RE, entonces (L1 $\wedge$ L2) $\in$ CO-RE}
\end{enumerate}



% -----------------------------------------------------
% Second problem
% -----------------------------------------------------

\item This is a problem about a double integral!

$$\int_{y=0}^{y=6} \int_{x=0}^{x=6-y} f(x,y) \ dx \ dy = 8/3.$$

or if you don't want it displayed, $\int_{y=0}^{y=6} \int_{x=0}^{x=6-y} f(x,y) \ dx \ dy = 8/3,$ or perhaps you prefer $\displaystyle \int_{y=0}^{y=6} \int_{x=0}^{x=6-y} f(x,y) \ dx \ dy = 8/3$?

\item Let's take partial derivatives a different way! 

$$\frac{\partial}{\partial x} xy = y.$$

If you need a particular symbol and don't know the code for it, try \url{detexify.kirelabs.org}.

\end{enumerate}

% ---------------------------------------------------
% Anything after the \end{document} will be ignored by the typesetting.
% ----------------------------------------------------

\end{document}

